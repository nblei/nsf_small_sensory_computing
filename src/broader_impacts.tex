\textbf{Artifacts.} The EarBench and NoseBench benchmark suites will be
released as open-source software to the community to further academic and
commercial development of sensory computing systems based on CPU, GPU, DSP, and
spatial computer architectures. We will also prototype custom earable
and odor processing computers.

\textbf{Workforce development.}  This work will train several graduate and
undergraduate students an in interdisciplinary area at the intersection of
architecture, application software, and hardware design at both logical and
physical levels.  Graduate students involved in the project will get hands-on
experience with computer architecture, physical design, hardware verification,
and software engineering.

\textbf{Interactions with Industry.}  The PI has active collaborations with
Apple Inc. and Intel, enabling the prototyping of an earable computer, and
providing graduate and undergraduate students with mentorship and advising from
industry professionals in hardware design and verification.  The hope is to
develop these collaborations further to enable transfer and potential
commercialization of the earable and odor processing designs developed in this
research.

\textbf{Undergraduate research.}  Undergraduate researchers working with PI
Kumar have recently published in premiere computer architecture conferences,
including ISCA, HPCA, and DAC.  The HPCA paper was nominated for a Best Paper
Award, and DAC and ISCA papers received significant media coverage.  Several
undergraduate researchers have gone on to join graduate studies at the
University of Illinois, the Massachusetts Institute of Technology, and Carnegie
Mellon University, among others.  The nature of the prototyping aspect of this
project, as well as the large scope of the benchmarking aspect of this project,
will help us engage numerous undergraduate students in research, including the
design and verification of novel hardware architectures, and the development of
novel quality of results metrics for the proposed benchmark suite.

\textbf{Curriculum development activities.}  PI Kumar was recently awarded the
Ronald W. Pratt Faculty Outstanding Teaching Award and the Stanley H. Pierce
Faculty Award at the University of Illinois for, among other things, innovative
teaching of project-based courses in computer architecture and system design.
He will blend in a discussion of earable computing in his graduate and
undergraduate computer architecture classes.  He will also list sensory
computing related projects as project options in the classes, while making
available the EarBench and NoseBench suites and earable design space
exploration tools, developed by this research, to his students.

