
We ask the question: What kind of programmable hardware will be needed to
support earable computing in the future?  Today's programmable hardware
platforms for earables~\cite{MT25, nxplpc, is2062, quicklogic, nuvoton} need to
support only limited computation and sensing and typically consist of either
microcontrollers or DSPs or both, running at low to moderate frequencies. The
choice of the earable microcontrollers (Cortex M4 and M7 cores are popular due
to their DSP extensions) and DSPs (Tensilica HiFi DSPs are popular due to their
relatively low power), their number, and their operating frequencies are
limited by the capacity limitations of today's earable batteries (e.g., AirPods
Pro uses a \SI{45.4}{\milli\ampere\hour}, \SI{3.7}{\volt}
battery~\cite{airpodsprobattery}).

Analogously, we ask the question: What kind of programmable hardware will be needed to
support olfactory computing in the future?
Previous work on \olfc{} has largely focused on sensing applications. A lot of
work exists on commercial and academic e-noses for specific tasks, such as wine
classification~\cite{buratti2004characterization}, evaluating readiness of blue
cheeses~\cite{valente2018cheeses}, assessing seafood freshness~\cite{grassi2022seafood}, and
diagnosing diseases in humans and animals~\cite{gardner2000electronic,
binson2021discrimination, va2021noninvasive, d2010investigation}.  While many
e-noses use electrical resistive sensors, others use gas chromatography-mass
spectrometry~\cite{electronic_sensor_technology, alpha_2018, pan2014early},
which can provide high fidelity chemical detection, but requires a large,
expensive, and power-hungry system, and is thus not suitable for many \olfc{}
applications such as a smart-bandage.
%Fig.~\ref{fig:man_vs_machine} shows the similarities between \olfc{} and
%human olfaction.
Other than sensing, some recent work has begun to re-look at the \textit{odor
synthesis} problem.%\footnote{only 60 years after the disastrous introduction of
%Smell-O-Vision~\cite{smith_kiger_2006}}
Odor synthesis is the creation of
chemicals or chemical mixtures which are perceived to have a certain odor.
There are several contemporary attempts at commercializing odor synthesis,
including a digital scent diffuser - AromaShooter~\cite{aromajoin_corporation},
and wearable devices such as jewelry~\cite{tillotson2006scent} and
headbands~\cite{amores2018promoting}. Odor synthesis has also received interest
within the context of extended reality.  VR SCENT~\cite{olorama_technology}
augments a standard VR system with odor synthesis to assess neuorological
trauma, while The Smell Engine~\cite{bahremand2022smell} is a general odor
synthesis pipeline for VR applications.
No prior work addresses
the organization and design of the olfactory processor itself.


To exploit these opportunities in earable and olfactory processing:
\iffalse
we propose to develop
tools for computer architects, hardware designers, software engineers, and
application developers to better understand sensory computing's software and
hardware ecosystems.  Specifically, the proposed research has two main thrusts,
the first related to \textit{earable computing} and the second to \textit{\olfc}.
The research goals along both thrusts parallel one another:
\fi

1. We will design and publish (open source) benchmark suites for earable and
\olfc{}.  
%Researchers and industry require a benchmark suites
%consisting of representative applications and associated input data.  
Benchmarks
will be written in the C programming language to ensure portability across
existing hardware.  All benchmarks will be published as scalar
and multithreaded workloads (using OpenMP API) to enable benchmarking on single
core, multicore, SMT, and SIMT architectures.

2. We will explore computer architectures for  earable computing and \olfc{}. The exploration will look at both heterogeneous solutions where a system-on-chip targets each computational characteristic with a different component ({\em accelerator}) as well as monolithic solutions where a single processor targets all characteristics. Both classes of solutions could be interesting since both cost and efficiency are concerns.

3. We will prototype an earable computing processor and an \olfc{}
processor, and evaluate the prototypes on the benchmark suites.  This will
give researchers a rigorous performance baseline to compare against, and will
also enable potential technology transfer and commercialization of our sensory
processors.

4. Both earable computing and \olfc{}  may be amenable to computational offloading wherein the
the processor itself is mainly used to collect sensor data (e.g.,
microphones, odor sensors, etc.) and drive actuators (e.g., speakers, lights,
microfluidics, antennae, etc.) while the majority of the computational
requirements of the applications are satisfied on other devices, such
as other wearable or personal devices (smart watch, smart phone, etc.), or
offloaded to cloud or edge server nodes. We will develop offloading-based implementations of applications for different earable and olfactory processor architectures.
