To exploit these opportunities in sensory processing, we propose to develop
tools for computer architects, hardware designers, software engineers, and
application developers to better understand sensory computing's software and
hardware ecosystems.  Specifically, the proposed research has two main thrusts,
the first related to \textit{earable computing} and the second to \textit{\olfc}.
The research goals along both thrusts parallel one another:

1. We will design and publish (open source) benchamark suites for earable and
\olfc{}.  Researchers and industry require a benchmark suites
consisting of representative applications and associated input data.  Benchmarks
will be written in the C programming language to ensure portability across
existing hardware.  All benchmarks will be published as scalar
and multithreaded workloads (using OpenMP API) to enable benchmarking on single
core, multicore, SMT, and SIMT architectures.

2. We will prototype an earable computing processor and an \olfc{}
processor, and evaluate the prototypes on the benchmark suites.  This will
give researchers a rigorous performance baseline to compare against, and will
also enable potential technology transfer and commercialization of our sensory
processors.
