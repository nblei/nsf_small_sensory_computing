Research objectives for earable computing are as follows:

\begin{enumerate}
\item Build EarBench --- an earable computing benchmark suite --- by porting
    all benchmark code to C and OpenMP, designing representative inputs for
    each benchmark, and providing baselines on existing systems; enable
    EarBench evaluations for heterogeneous systems --- both heterogeneous
    SoCs, as well as multi-chip and even multi-device systems.  EarBench
    --- software, data, \& build scripts --- as well as executable binaries
    for major platforms --- will be published under an open source license in
    order to enable commercial and academic research in earable computing.

    EarBench will consist of a diverse and representative set of current and
    future earable computing applications.  Data inputs will be drawn from
    real world inputs (e.g., measured microphone, accelerometer, \& ECG data),
    popular queries to current question-answering systems (e.g., Alexa,
    Google Assistant, etc.).

    EarBench will be published with a CMake based build system.  As CMake
    is an open-source, cross platform build automation tool targeting all
    major operating systems, and since nearly all computer architectures
    have open source or proprietary C compilers, researchers will be able
    to get EarBench running on a wide variety of systems quickly and easily.
    For major architectures (e.g., ARMv7, ARMv8, x86\_64, i686, etc.),
    we will provide prebuilt EarBench binaries.
    
\item
    Prototype, and test an earable computer in in \(\leq
    \SI{65}{\nano\meter}\) technology.
    The candidate design, SpEaC (see Sec.~\ref{ssec:prelim1}), shows
    significant performance and energy improvement over existing hardware
    systems used in today's earable computers.

    Prototyping subobjectives include physical design of SpEaC, including
        design and placement of multiple power and clock domains, integration
        of SpEaC with commercially licensable SRAMs and other memories, design
        for test, chip tape-out, and packaging.  Multiple power domains are
        used as SpEaC is a heterogeneous system, and thus power gating can be
        used to reduce static power consumption. Similarly, clock and voltage
        scaling can be used minimize energy or power consumption when one or
        more of SpEaC's components has performance slack available.  Although
        SpEaC's architecture and preliminary design are complete, architectural
        research determining the optimal choice of operating points is
        required.  PI Kumar is experienced in digital design, chip design,
        tape-out and testing, and runs an undergraduate course in which
        students design and tape-out processors which are manufactured by TSMC
        and then tested at UIUC campus.  His research group also participates
        in Intel's Chip Design Challenge, in manufacturing RISC-V processors in
        Intel's \SI{16}{\nano\meter} technology.

    Research subobjectives relating to testing include earable system design
        and PCB design, porting EarBench applications to SpEaC, and collecting
        measured EarBench results.  System design will account for various
        earable sensors (e.g., microphone arrays, inertial motion units),
        actuators (e.g., speakers, antennae), and hardware debug systems.
        Porting EarBench applications to SpEaC requires annotating the EarBench
        C code with pragmas which enable SpEaC's compiler to automatically
        generate configurations and data streams for its spatial accelerator.


\item Analyze SpEaC and existing earable hardware in a heterogeneous, networked
    computational setting, in which computation can be offloaded from the
    earable device onto a number of other devices, including smart watches,
    smart phones, and high perfromance cloud systems.  determine the 
\end{enumerate}
