Research objectives for \olfc{} are as follows:

\begin{enumerate}
\item Extend the representative olfactory computing applications (Sec~\ref{sec:tract2})
    into a bona fide olfactory computing benchmark suite --- NoseBench --- by
    porting all benchmark code to C and OpenMP, designing representative inputs
    for each benchmark, and providing baselines on existing systems, including
    Ahromaa; enable evaluations for heterogeneous systems --- both
    heterogeneous SoCs, as well as multi-chip and even multi-device systems.
    NoseBench --- software, data, \& build scripts --- as well as
    executable binaries for major platforms --- will be published under
    an open source license in order to enable commercial and academic
    research into \olfc{}.

    NoseBench will consist of a representative set of current and future
    \olfc{} applications.  Data and inputs will be drawn from published real
    world samples.
    
    NoseBench will be published with a CMake based build system.  As CMake
    is an open-source, cross platform build automation tool targeting all
    major operating systems, and since nearly all computer architectures
    have open source or proprietary C compilers, researchers will be able
    to get NoseBench running on a wide variety of systems quickly and easily.
    For major architectures (e.g., ARMv7, ARMv8, x86\_64, i686, etc.),
    we will provide prebuilt NoseBench binaries.
\item
    Rapid development of small and ultra-low power odor sensors means that, for
    the first time, computing power consumption may be the limiting factor
    in olfactory or chemical processing, rather than sensor power
    consumption.  In light of this development, we will perform an exploration
    of the architectural design space in order to identify good architectures
    for \olfc{} systems targeting wearable and distributed sensor networking
    olfactory applications.  Many ultra-low power candidate architectures
    currently exist, including ultra-low power microcontrollers~\cite{} and GPUs~\cite{},
    as well as \(<\si{\milli\watt}\) spatial architectures~\cite{}.
    Preliminary results analyzing olfactory applications
    and their computational requirements (Sec~\ref{ssec:prelim2}) show that,
    for many applications, performance requirements are lax, and absolute
    computational and memory requirements are low.  This suggests that the
    architectural design space considered must go beyond looking at just
    computer organization and design, but must encompass memory design and
    operating points.

    Due to lax performance requirements and low computational requirements,
    \olfc{} is likely a strong candidate for near or sub-threshold
    computing~\cite{}.  Thus the design space exploration will include
    ammenability to and impact of voltage scaling into the near and sub-threshold
    regions.  Although this type of work has been done for CPU and GPU based
    architectures, it has, to the best of our knowledge, never been done for
    reconfigurable spatial architectures.
    
    As many \olfc{} applications require very small amounts of data memory,
        memory design will be an important aspect of the design space
        exploration.  The size, geometry, and number of SRAM banks will be
        explored.  Additionally, since some applications require trivial
        amounts of data memory, we will also consider memories composed wholy
        or partially out of latches and flip-flops, rather than SRAM cells.
        Although SRAM cells are typically smaller and consume less power than
        flip-flops and latches at nominal voltage, SRAM cells are generally not
        ammenable to voltage scaling.  Thus, at low voltages, latch and
        flip-flop based memories may consume less power than an equivalently
        sized SRAM.

\item Analyze Ahromaa and existing olfactory computing hardware in a heterogeneous, networked
     computational setting, in which computation can be offloaded from the earable
     device onto a number of other devices, including smart watches, smart phones,
     and high perfromance cloud systems;
\end{enumerate}

